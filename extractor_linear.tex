%\newcommand\Ext{\mathsf{Ext}}
\section{Introduction}

\begin{definition}[Useful definition of total variation distance.]
  The distance between two distributions $f(X)$ and $U_q$ over $[q]$ is given by 
  \begin{align}
    \max_{T\subset [q]} |\Pr[f(X)\in T] - \Pr[U_q\in T]|.
    \label{eq:tv}
  \end{align}
\end{definition}
\begin{definition}[Useful chernoff bound]
  Let $X_1,\dots,X_t$ be random variables on $[0,1]$, and $X=\sum_{i}^{} X_i/t$. Then
  \begin{align}
    \Pr[|X-\mu|\ge \varepsilon]\l2 2\exp(-t\varepsilon^2/4).
    \label{}
  \end{align}
\end{definition}

Useful bounds for extractors: \url{https://homes.cs.washington.edu/~anuprao/pubs/lowweight.pdf} (Theorem 2.10, Cor 2.11, Cor 2.12, Cor 2.13)


\url{https://people.seas.harvard.edu/~salil/pseudorandomness/extractors.pdf}
\begin{definition}
  An function $\Ext:\{0,1\}^n\times \{0,1\}^d\to\{0,1\}^m$ is a \emph{strong seeded extractor} for min-entropy $k$ and error $\varepsilon$ if for every min-entropy $k$ source $X$,
  \begin{align}
    \Pr_{u\sim U_d}\left[ |\Ext(X,u) - U_m|\le \varepsilon \right]\ge 1-\varepsilon.
    \label{}
  \end{align}

  An extractor is \emph{linear} if $\Ext(\cdot,u)$ is a lienar function over $GF(2)$ for every $u\in\{0,1\}^d$.
\end{definition}

\begin{proposition}[Thm 6.12 in Vadhan, optimal output length]
  There is an extractor $\Ext:\{0,1\}^n\to\{0,1\}^m$ with $m=k-2\log(1/\varepsilon)-O(1)$, where $k$ is the min-entropy requirement.

  Supposedly there is an impossibility result that  $m\le k-2\log(1/\varepsilon)+O(1)$
\end{proposition}
\begin{proof}
  Take a random extractor. Let $X$ be a min-entropy $k$ distribution. \ryl{I think this proof needs some work...you need some weighted sum for the Chernoff bound. Oh. Lemma 6.10 of Vadhan. Every $k$-source is a convex combination of flat $k$-sources (if $2^k$ is an integer), so it suffices to show something is an extractor for flat $k$-sources.}
  The probability that \eqref{eq:tv} is violated for $f(\cdot)=\Ext(\cdot)$ is $2^{-\Omega(2^k\varepsilon^2)}$ by Chernoff bound: for each $x\in\Supp(X)$ (there are at least $2^k$), the probability that $\Ext(x)\in T$ is $\mu(T)$, so the mean is $\mu(T)$ and then you use Chernoff bound to show concentration with bound $2^{-\Omega(K\varepsilon^2)}$, and union bound over the $2^q=2^{2^m}$ possible subsets.
\end{proof}
\begin{theorem}[Thm 6.14 in Vadhan]
  For every $n$, $k\in[0,n]$ and $\varepsilon > 0$, there exists a $(k,\varepsilon)$-extractor $\Ext : \{0,1\}^n \times  \{0,1\}^d \to \{0,1\}^m$ achieving $m \le k - 2\log(1/\varepsilon) - O(1)$ and $d \ge \log(n - k) + 2\log(1/\varepsilon) + O(1)$. \ryl{In Vadhan, the definition is a bit different: there is an extra $+d$ in $m$ which includes the seed bits}
  Both $d$ and $m$ are optimal up to the $O(1)$.
\end{theorem}
\begin{proof}
  Random construction again.
\end{proof}


\begin{definition}
  Given a rate $R$ code $C$ given by encoding function $\Enc:[q]^{Rn}\to C\subset[q]^n$, the corresponding extractor $\Ext:[q]^{Rn}\times [n]\mapsto [q]$ given by $\Ext(x,y) = \Enc(x)_y$.
\end{definition}

\begin{question}
  Let $q=2^m$.
  Let $\Enc:[q]^{Rn}\to[q]^n$ be a random linear code, and $\Ext:[q]^{Rn}\times [n]\to [q]$ be the corresponding extractor.
\begin{enumerate}
  \item Is this a linear extractor? YES. It is $\mathbb{F}_q$ linear, and when we view it as a map of binary bits you can see it as $\mathbb{F}_2$ linear. We can identify each $\mathbb{F}_q$ element by a vector (a polynomial in $\mathbb{F}_2[t]/p(t)$ where $p$ is a degree-$m$ irreducible. Here $q=2^m$), and then multiplying by any $\alpha\in\mathbb{F}_q$ element is an $m\times m$ $\mathbb{F}_2$-matrix multiplication (the columns are the images of $1,t,t^2,\dots,t^{m-1}\in\mathbb{F}_q$ under multiplication by $\alpha$).

  \item Is this a \emph{good} extractor? This is the question.

    Is there a code with $q=\poly(1/\varepsilon)$ such that among any $L=\poly q$ codewords $\mathcal{L}$ and all $i\in[n]$,
    \begin{align}
      \E_{i\sim [n]}\max_{T\subset[q]}\abs{\Pr_{c\sim \mathcal{L}}\left[ c_i\in T \right] - \Pr\left[ U_q\in T \right] }\le \varepsilon
      \label{}
    \end{align}

    This would give an extractor with seed length $\log n$ (but our $Rn$ is their $n$).
    I guess if you want to be list-decodable to radius $1-1/q-\varepsilon$, then your rate can be at best $1/\varepsilon^2$, so then your seed length is actually $\log n/\varepsilon^2 = \log n +2\log 1/\varepsilon$, so that's where the optimal seed length comes from. So really we want optimal rate, and we're happy with ``good enough'' alphabet size and list size.

    This would give min-entropy $k$ and output length $q$ both in $\Theta(\log 1/\varepsilon)$ time.
    Obviously, $L=2^k\ge q/\varepsilon^2$ makes sense, because with a smaller list size, some coordinates with be unsupported.

    \ryl{I guess $q=1/\varepsilon$ is the actual parameter regime that we care about, not $q=2^{1/\varepsilon}$.}

    \ryl{I'm confused, why don't we just look at each coordinate separately and union bound over all $n$ coordinates?}
    \ryl{Wait this is obviously impossible since on any coordinate you can obviously find $L$ coordinates that all have the same value. Oh. It's an average over all coordinates. Yeah so it is like a list recovery condition.}
\end{enumerate}
\end{question}
